\documentclass[11pt]{article}
\usepackage{enumitem}
\usepackage{amsmath,amsthm,amssymb}
\usepackage{color}

\begin{document}
\title{Homework 4\\--\\\large CS270 Fall 2020}
\date{}
\author{Brennen Green}
\maketitle

\section{Practice Problems, 3.24, 3.25, 3.28, 3.31 3.32, 3.35}
\begin{enumerate}
	\item[(3.24)] \begin{verbatim}
		short loop_while(short a, short b) {
		    short result = 0;
		    while (a > b) {
		        result = b + a ;
		        a = a - 1;
		    }
		    return result;
		}
	\end{verbatim}
	\item[(3.25)] \begin{verbatim}
		long loop_while(short a, short b) {
		    long result = b;
		    while (b>0) {
		        result *= a;
		        b = b-a;
		    }
		    return result;
		}
	\end{verbatim}
	
	\item[(3.28)] \begin{verbatim}
		long test_two(unsigned test x) {
			long val = 0;
			for (int i = 64; i != 0; i--) {
			    val =  (val << 1) | (x & 0x1);
			    x >>= 1;
			}
			return val;
		}
	\end{verbatim}

	\item[(3.31)] \begin{verbatim}
		void switcher(long a, long b, long c, long *dest) {
		    long val;
		    switch (a) {
		    case 5:
		        c = b ^ 15;
		    case 0:
		        val = c + 112;
		        break;
		    case 2:
		    case 7:
		        val = (c + b) << 2;
		        break;
		    case 4:
		        val = a;
		        break;
		    default:
		        val = b;
		    }
		    *dest = val;
		}
	\end{verbatim}
	\end{enumerate}
	(3.32)
\begin{center}
		\begin{tabular} {|| c | c | c | c | c | c | c | c | c ||}
			\hline
			Label & PC & Instruction & \%rdi & \%rsi & \%rax &\%rsp & *\%rsp & Description \\ [0.5ex] 
			\hline
			\hline
			M1 & 0x400560 & callq & 10 & -- & -- & 0x7fffffffe820 & -- & Call first(10) \\
			F1 & 0x400548 & lea & 10 & 11 & -- & 0x7fffffffe818 & 0x400565 & x+1\\
			F2 & 0x40054c & sub & 9 & 11 & -- & 0x7fffffffe818 &  0x400565 & x-1 \\
			F3 & 0x400550 & callq & 9 & 11 & -- & 0x7fffffffe818 &  0x400565 & Call last(x-1, x+1) \\
			L1 & 0x400540 & mov & 9 & 11 & 8 & 0x7fffffffe810 &  0x400555 & u \\
			L2 & 0x400543 & imul & 9 & 11 & 80 & 0x7fffffffe810 &  0x400555 & u * v\\
			L3 & 0x400547 & retq & 9 & 11 & 80 & 0x7fffffffe810 &  0x400555 & Return \\
			F4 & 0x400555 & repz & 9 & 11 & 80 & 0x7fffffffe818 &  0x400565 & Return \\
			M2 & 0x400565 & mov & 9 & 11 & 80 & 0x7fffffffe820 & -- & Resume \\
			\hline 

		\end{tabular}
\end{center}
	\newpage
	(3.35)
	\begin{verbatim}
		long rfun(unsigned long x) {
		    if (x == 0) return 0;
		    unsigned long nx = x >> 2;
		    long rv = rfun(nx);
		    return x + rv;
		}
	\end{verbatim}

\section{Homework Problem 3.60}
\begin {verbatim}
long loop(long x, long n) {
    long result = 0;
    long mask;
    for (mask = 1; mask != 0; mask = mask << (n & 0xFF)) {
	        results |= (x&mask);
	    }
    return result;
}
\end{verbatim}
\begin{enumerate}
	\item[A.] x is in \%rdi and \%r8, n is in \%esi and \%ecx, result is in \%rax, mask is in \%rdx
	\item[B.] result = 0, mask = 1
	\item[C.] mask is not equal to zero
	\item[D.] by shifting the lower 8 bits of n to the left
	\item[E.] by the bitwise or operation with (x \& mask)
\end{enumerate}
\end{document}

