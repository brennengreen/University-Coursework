\documentclass[11pt]{article}
\usepackage{enumitem}
\usepackage{amsmath,amsthm,amssymb}
\usepackage{color}

\begin{document}
\title{Homework 1\\--\\\large CS270 Fall 2020}
\date{}
\author{Brennen Green}
\maketitle

\section{Practice Problems 2.17, 2.19, 2.24, 2.29, 2.47}
    \begin{enumerate}
        \item[(2.17)]
            \begin{tabular}{||c | c | c | c||} 
            \hline
            Hexadecimal & Binary & $B2U_4(x)$ & $B2T_4(x)$ \\ [0.5ex] 
            \hline\hline
            0xA & 1010 & $2^3 + 2^1 = 10$ & -$2^3 + 2^1 = -6$ \\ 
            \hline
            0x1 & 0001 & $2^0 = 1$ & $ 2^0 = 1 $ \\
            \hline
            0xB & 1011 & $2^3 + 2^1 + 2^0 = 11 $ & $-2^3 + 2^1 + 2^0 = -5$ \\
            \hline
            0x2 & 0010 & $ 2^1 = 2 $ & $ 2^1 = 2 $ \\
            \hline
            0x7 & 0111 & $ 2^2 + 2^1 + 2^0 = 7 $ & $ 2^2 + 2^1 + 2^0 = 7 $ \\
            \hline
            0xC & 1100 & $ 2^3 + 2^2 = 12 $ & $ -2^3 + 2^2 = -4 $ \\ 
            \hline
            \end{tabular}
           
        \item[(2.19)]
            \begin{tabular}{||c | c||} 
            \hline
            x & $T2U_4(x)$ \\ [0.5ex] 
            \hline\hline
            -1 & 15  \\ 
            \hline
            -5 & 11 \\
            \hline
            -6 & 10 \\
            \hline
            -4 & 12 \\
            \hline
            1 & 1 \\
            \hline
            8 & 8 \\ 
            \hline
            \end{tabular}
        \item[(2.24)]
            \begin{tabular}{||c | c || c | c || c | c||} 
            \hline
            Original & Truncated & Original & Truncated & Original & Truncated \\ [0.5ex] 
            \hline\hline
            1 & 1  & 1 & 1 & 1 & 1\\ 
            \hline
            3 & 3 & 3 & 3 & 3 & 3\\
            \hline
            5 & 5 & 5 & 5 & 5 & 5\\
            \hline
            C & 4 & 12 & 4 & -4 & -4\\
            \hline
            E & 6 & 14 & 6 & -2 & -2\\
            \hline
            \end{tabular}
        \item[(2.29)]
            \begin{tabular}{||c | c | c | c | c ||} 
            \hline
            x & y & x + y & $x +^t_5 y$ & Case \\ [0.5ex] 
            \hline\hline
            10100 & 10001  & 100101 & 00101 &  1\\ 
            \hline
            11000 & 11000 & 110000 & 10000 &  2\\
            \hline
            10111 & 01000 & 111111 & 11111 &  2\\
            \hline
            00010 & 00101 & 000111 & 00111 & 3\\
            \hline
            01100 & 00100 & 010000 & 10000 &  4\\
            \hline
            \end{tabular}
        \item[(2.47)]
            \begin{tabular}{||c | c | c | c | c | c | c | c | c | c | c||} 
            \hline
            Bits & e & E & $2^E$ & $f$ & M & $2^E \times M$ & V & Decimal \\ [0.5ex] 
            \hline\hline
            0 00 00 & 0 & -1 & 1/2 & $\frac{0}{4}$ & $\frac{4}{4}$ & $\frac{1}{8}$ & $\frac{1}{8}$ & 0.125 \\
            \hline
            0 00 01 & 0 & -1 & 1/2 & $\frac{1}{4}$ & $\frac{5}{4}$ & $\frac{5}{8}$ & $\frac{5}{8}$ & 0.625 \\
            \hline
            0 00 10 & 0 & -1 & 1/2 & $\frac{1}{2}$ & $\frac{3}{2}$ & $\frac{3}{4}$ & $\frac{3}{4}$ & 0.75 \\
            \hline
            0 00 11 & 0 & -1 & 1/2 & $\frac{3}{4}$ & $\frac{7}{4}$ & $\frac{7}{8}$ & $\frac{7}{8}$ & 0.875 \\
            \hline
            0 01 00 & 1 & 0 & 1 & $\frac{0}{4}$ & $\frac{4}{4}$ & $\frac{4}{4}$ & $\frac{1}{1}$ & 1.00 \\
            \hline
            0 01 01 & 1 & 0 & 1 & $\frac{1}{4}$ & $\frac{5}{4}$ & $\frac{5}{4}$ & $\frac{5}{4}$ & 1.25 \\
            \hline
            0 01 10 & 1 & 0 & 1 & $\frac{1}{2}$ & $\frac{3}{2}$ & $\frac{3}{2}$ & $\frac{3}{2}$ & 1.50 \\
            \hline
            0 01 11 & 1 & 0 & 1 & $\frac{3}{4}$ & $\frac{7}{4}$ & $\frac{7}{4}$ & $\frac{7}{4}$ & 1.75 \\
            \hline
            0 10 00 & 2 & 1 & 2 & $\frac{0}{4}$ & $\frac{4}{4}$ & $\frac{8}{4}$ & $\frac{2}{1}$ & 2.00 \\
            \hline
            0 10 01 & 2 & 1 & 2 & $\frac{1}{4}$ & $\frac{5}{4}$ & $\frac{10}{4}$ & $\frac{5}{2}$ & 2.50 \\
            \hline
            0 10 10 & 2 & 1 & 2 & $\frac{1}{2}$ & $\frac{3}{2}$ & $\frac{6}{2}$ & $\frac{3}{1}$ & 3.00 \\
            \hline
            0 10 11 & 2 & 1 & 2 & $\frac{3}{4}$ & $\frac{7}{4}$ & $\frac{14}{4}$ & $\frac{7}{2}$ & 3.50 \\
            \hline
            0 11 00 & 3 & 2 & 4 & $\frac{0}{4}$ & $\frac{4}{4}$ & $\frac{12}{4}$ & $\frac{3}{1}$ & 3.00 \\
            \hline
            0 11 01 & 3 & 2 & 4 & $\frac{1}{4}$ & $\frac{5}{4}$ & $\frac{15}{4}$ & $\frac{15}{4}$ & 3.75 \\
            \hline
            0 11 10 & 3 & 2 & 4 & $\frac{2}{4}$ & $\frac{6}{4}$ & $\frac{18}{4}$ & $\frac{9}{2}$ & 4.50 \\
            \hline
            0 11 11 & 3 & 2 & 4 & $\frac{3}{4}$ & $\frac{7}{4}$ & $\frac{21}{4}$ & $\frac{21}{4}$ & 5.25 \\
            \hline
            \end{tabular}      
    \end{enumerate}


\section{Homework Problems 2.77}
    \begin{enumerate}
        \item K = 17 \newline
            $$17 \times x$$
            $$ 16 \times x + x $$
            $$ (n << 4) + x $$
        \item K = -7
            $$ -7 \times x $$
            $$ x - (8 \times x) $$
            $$ x - (x << 3) $$
        \item K = 60
            $$ 60 \times x $$
            $$ (64 \times x) - (4 \times x) $$
            $$ (x << 6) - (x << 2) $$
        \item K = -112
            $$ -112 \times x $$
            $$ (16 \times x) - (128 \times x) $$
            $$ (x << 4) - (x << 7) $$
    \end{enumerate}

\section{L2\_show-bytes.c}
    The number 40506 is equivalent to the binary number 1001 1110 0011 1010
    this would be represented in Hexadecimal as 0x9E3A this means the representation
    in big-endian would be similar to the following
    \newline\newline
    \begin{tabular}{c | c}
        \hline
        Address & Byte \\
        \hline
        0x7fff73c4f1cc & 9E\\
        0x7fff73c4f1cd & 3A\\
        0x7fff73c4f1ce & . . .\\
        0x7fff73c4f1cf & . . .\\
        
    \end{tabular}


\end{document}

