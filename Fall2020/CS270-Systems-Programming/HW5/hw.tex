\documentclass[11pt]{article}
\usepackage{enumitem}
\usepackage{amsmath,amsthm,amssymb}
\usepackage{color}

\begin{document}
\title{Homework 5\\--\\\large CS270 Fall 2020}
\date{}
\author{Brennen Green}
\maketitle

\section{Practice Problems, 3.38, 3.41, 3.45, 3.48}
\begin{enumerate}
	\item[3.38] M = 5, N = 7
	\item[3.41]
	\begin{verbatim}
		void st_init(struct test *st) {
		    st->s.y = st->s.x;
		    st->p = &(st->s.y);
		    st->next = st;
		}
	\end{verbatim} 
	\begin{enumerate}
		\item[A.] 8 Bytes for p, 2 bytes for s.x, 2 for s.y, 8 bytes for s->next
		\item[B.] 20 Bytes
		\item[C.] (see code) 
	\end{enumerate} 
	\item[3.45]
	\begin{enumerate}
		\item[A.] *a = 0 bytes, b = 8 bytes, c = 12 bytes, d = 16 bytes, e = 24 bytes, f = 32 bytes,
		g = 40 bytes, *h = 48 bytes  
		\item[B.] 56 bytes
		\item[C.] *a, *h, f, e, b, g, d, c
	\end{enumerate}
	\item[3.48]
	\begin{enumerate}
		\item[A.] Unprotected: v = \%rsp + 24, buf = \%rsp Protect: canary = offset 40, v = offset 8, buf = offset 16
		\item[B.] The positioning of v allows the programmer to avoid a overflow of buf that might conflict with v
	\end{enumerate}
\end{enumerate}
\newpage
\section{Homework Problem 3.68}
\begin{verbatim}
	A = 9
	B = 5
\end{verbatim}
\end{document}

void setVal(str1 *p, str2 *q)
p in %rdi, q in %rsi
setVal:
 movslq 8(%rsi), %rax // rax = 8 + *q
 addq 32(%rsi), %rax // rax = 32 + *q
 movq %rax, 184(%rdi) // rax = 184 + *p
 ret