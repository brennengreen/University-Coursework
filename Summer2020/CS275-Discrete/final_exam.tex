\documentclass[11pt]{article}
\usepackage{enumitem}
\usepackage{amsmath,amsthm,amssymb}
\usepackage{color}
\usepackage{tikz}

% Tikz settings optimized for causal graphs.
% Just copy-paste this part
\usetikzlibrary{shapes,decorations,arrows,calc,arrows.meta,fit,positioning}
\tikzset{
    -Latex,auto,node distance =1 cm and 1 cm,semithick,
    state/.style ={ellipse, draw, minimum width = 0.7 cm},
    point/.style = {circle, draw, inner sep=0.04cm,fill,node contents={}},
    bidirected/.style={Latex-Latex,dashed},
    el/.style = {inner sep=2pt, align=left, sloped}
}

\begin{document}
\title{CS275 Final Exam}
\author{Brennen Green}
\maketitle


\begin{enumerate}
    \item The coefficient of $x^6$ in the expansion of $(7x+3y)^{11} = \frac{11!}{5!(11-5)!}(7x)^6(3y)^5 = 13207982634$ (coefficient only)
    \item In this question you stated there was a flush (five cards of the same suit) but then defined a straight (five cards with consecutive types)
        I did not know which to use for my answer so I will give two parts using both the number of flushes + straight flushes and the number of straights + straight flushes.
        \begin{enumerate}
            \item The Question Using Flushes: \newline
                Possible flushes (including straight flushes):$ 4 * C(13, 5) = 4 * 1287 = 5,148$ possible 5-card hands
            \item The Question Using Straights: \newline
                Possible straights (including straight flushes):$ 10 * 4^5 = 10,240$ possible 5-card hands
        \end{enumerate}
    \item $2^9 - 2C(9,0) - 2C(9,1) - 2C(9,2) - C(9,3) = 336$ bit strings
    \item E and F are not indepedent. In order for two events to be independent then
        P(E)P(F) = P(E$\cap$F). P(E) = $\frac{15}{32}$ P(F) = $\frac{1}{5}$. P(E$\cap$F) = $\frac{1}{4}$
        Since in this scenario $P(E)*P(F) = \frac{15}{160} \neq \frac{1}{4}$ the events E and F are not independent
    \item $\frac{C(4, 2) + C(3, 2) + C(2,2)}{P(5,3)} = \frac{10}{60} = \frac{1}{6}$
    \item 2*Expected Value of Fair Die + Expected Value of Weighted Die = 2*3.5 + $\frac{11}{3}$ = $\frac{32}{3}$
    \newpage
    \item $r^2 - 7r - 12 = 0 \rightarrow r = \frac{7 \pm \sqrt{97}}{2}$\newline
        $a_n = \alpha_1(\frac{7 + \sqrt{97}}{2}) + \alpha_2 \frac{7 - \sqrt{97}}{2} \text{ AND } a_0 = 4\;a_1 = 7$\newline
        $\therefore \alpha_1 = \frac{194-7\sqrt{97}}{97} \;\; \alpha_2 = \frac{7\sqrt{97} + 194}{97}$ \newline
        $\therefore a_n = \frac{194-7\sqrt{97}}{97}(\frac{7 + \sqrt{97}}{2}) + \frac{7\sqrt{97} + 194}{97}(\frac{7 - \sqrt{97}}{2})$
    \item $$a_n = 1.05a_{n-1} + 1.10a_{n-2}$$
          $$a_0 = \$500 \; \; a_1 = \$525 $$
          $$a_2 = \$1,101.25 \; a_3 = \$1,733.81$$
          $$ \dots $$
          $$ a_{17} = \$2,930,701.11 $$
    \item $$ a_n = 8a_{n-1} + 10^{n-1} \; a_0 = 1$$
          $$ G(x) - a_0 = \sum_{k = 1}^{\infty}a_kx^k$$
          $$ = \sum_{k = 1}^{\infty}(8a_{k-1}+10^{k-1})x^k$$
          $$ = 8x\sum_{k = 1}^{\infty}a_{k-1}x^{k-1} + 10x^2\sum_{k = 1}^{\infty}10^{k-2}x^{k-2}$$
          $$ = 8xG(x) + \frac{10x^2}{1-10x} $$
          $$ G(x) = \frac{5x}{4(-10x + 1)} $$
    \item If $a, b, c \in \mathbb{Z}$ and $a | b \wedge b | c$ then $n_1$ and $n_2$ exists such that $a*n_1 = b$
        and $b*n_2 = c$ then $ (n_1 * n_2)a = c \rightarrow a | c$ and the relation is transitive
    \item In order for a relation to be transitive then if (a,b) and (b,c) exist in the relation, so must (a,c)
        That means this relation represented in the matrix is not transitive, there may be multiple examples in the
        matrix to prove this however we only need one to prove that it is not transitive. In the matrix (3,2) and (2,3)
        are both relations but (3,3) is not in the adjacency matrix.
    \item \begin{enumerate}
        \item A graph is reflexive if every element loops to itself
        \item A graph is irreflexive if no elements loop to themself
        \item A graph is symmetric if every edge of a directed graph has a compliment going in the opposite direction
        \item A graph is antisymmetric if no edge of a directed graph has a compliment in the opposite direction
        \item A graph is transitive if for ever pair of edge (x,y) and (y,z) there exists an edge (x,z)
    \end{enumerate}
\end{enumerate}

\end{document}

