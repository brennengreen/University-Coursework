\documentclass[12pt]{article}
\usepackage{enumitem}
\usepackage{amsmath,amsthm,amssymb}

\begin{document}
\title{Homework 3}
\author{Brennen Green}
\maketitle


\begin{enumerate}
    \item \begin{verbatim}Ch3.1 Exercise #5\end{verbatim}
        Start at the second element and sequentially iterate over the list of elements and comparing
        every element to the one before it. If an element is equal to the one before it, remove that
        element from the list.
    \item \begin{verbatim}Ch3.1  Exercise  #23\end{verbatim}
        A function is onto  if and only if $ f(a) = f(b) \rightarrow a = b $ so in order to check
        if a our two sets of integers and function are onto, we must ensure there is no case where
        $ f(a) = f(b) $ is true yet $ a \ne b $ \newline \newline
        So our algorithm must iterate over the image set $ b_i $, at every element we must iterate over 
        the pre-image $ a_j $ until we find an element where $ a_j = b_i $. If there is an element 
        $ b_i $ that does not have some matching element $ a_j $ then we return our algorithm as
        false. If our algorithm iterates over the entire image set with no interuption, the algorithm
        returns true. 
    \item \begin{verbatim}Ch3.2 Exercise #15\end{verbatim}
        If some function is $ \mathcal{O}(1) $ it is the same as saying that, in the worst case, a function
        will always produce some constant and never grow.
    \item \begin{verbatim}Ch3.2 Exercise #30\end{verbatim}
        If some function is $ \Omega(1) $ it is the same as saying that, in the best case, a function
        will always produce some constant and never grow.
        
    \newpage

    \item \begin{verbatim}Ch3.2 Exercise #31\end{verbatim}
        If some function is $ \Theta(1) $ it is the same as saying that, in the best case, a function
        will always produce some constant and never grow. Or put in other words, the function is both
        $ \Omega(1) $ AND $ \mathcal{O}(1) $
    \item \begin{verbatim}Ch3.3  Exercise  #28 \end{verbatim}
        \begin{enumerate}[label=\roman*]
            \item Since bubble sort makes $ \frac{n(n-1)}{2} $ (or ) comparisons with input n, we can plug 2n into this equation
            to find that with input 2n bubble sort makes $ 2n^2 - n $ comparisons. Dividing the amount of comparisons at 2n by
            the number of comparisons at n that this change in the number of comparisons was more than double and definitely
            significant.

            \item We have an identical process to compare the growth of comparisons for insertion sort. Since instertion sort
            also makes $ \frac{n(n-1)}{2} $ comparisons with an input of n, plugging 2n into this will show us that with an
            input of 2n insertion sort will need $ 2n^2 -n $ comparisons. Which is more than a doubling of comparisons and
            significant growth.
        \end{enumerate}
        These two results make sense as both functions have extremely similar computational complexities.
    \item \begin{verbatim}Ch3.4 Exercise #5 \end{verbatim}
        \begin{proof}
            \begin{align*}
                & a | b \wedge b | a \rightarrow (a = b) \vee (a = -b)  \\
                & a | b \rightarrow ac = b \wedge b | a \rightarrow bc = a && \text{(definition of division)} \\
                &  (bc)c = b  \\
                &  bc^2 = b \\
                &  c = \pm1  \\
                &  a(1) = b a(-1) = b  \\
                &  a = b \vee a = -b  \qedhere 
            \end{align*}
        \end{proof}

    \newpage

    \item \begin{verbatim}Ch3.4 Exercise #16\end{verbatim}
        \begin{enumerate}[label=\alph*]
            \item 1
            \item 4
            \item 3
            \item 9
        \end{enumerate}

    \item \begin{verbatim}Ch3.4 Exercise #24\end{verbatim}
        \begin{proof}
            \begin{align*}
                & \text{n is an odd integer} \rightarrow n^2 \equiv 1(mod 8) \\
                & n = 2k + 1 \\
                & 8 | n^2 - 1 \\
                & 8 | (2k + 1)^2 - 1 \\
                & 8 | 4k^2 + 4k \\
                & 8 | 4k(k + 1) \\
                & 8 | 4(2x)   && \text{(k(k+1) is even since k or k+1 is even)} \\
                & 8 | 8x \\
                & \therefore 8 | n^2 - 1 \\
                & \therefore n^2 \equiv 1(mod 8) && \text{(definition of congruency)} \qedhere 
            \end{align*}
        \end{proof}

    \item \begin{verbatim}Ch3.5 Exercise #4\end{verbatim}
        \begin{enumerate}[label=\alph*]
            \item $ 3 * 13 $ 
            \item $ 3 * 3 * 3 * 3 $
            \item $ 101 $
            \item $ 13 * 11 $
            \item $ 17 * 17 $
            \item $ 31 * 29 $

        \end{enumerate}
    
    \newpage

    \item \begin{verbatim} Ch3.5 Exercise #31 \end{verbatim}
        \begin{proof}
            \begin{align*}
                & \forall k \in \mathbb{Z}(6 | k*(k+1)*(k+2)) \\
                & 1.\;\; 2 | k*(k+1)*(k+2) && \text{(at least one number will be even)} \\
                & 2.\;\; 3 | k*(k+1)*(k+2) && \text{(every 3rd integer is divisible by 3)} \\
                & 3.\;\; 2 * 3 | k*(k+1)*(k+2) && \text{((a $|$ c) and (b $|$ c) then ab $|$ c)} \\
                & 4.\;\; \therefore 6 | k*(k+1)*(k+2) \qedhere
            \end{align*}
        \end{proof}
    
    \item \begin{verbatim} Ch3.6 Exercise #1 \end{verbatim}
        \begin{enumerate}[label=\alph*]
            \item 1110 0111
            \item 10 0011 0110 100
            \item 1 0111 1101 0110 1100
        \end{enumerate}

    \item \begin{verbatim} Ch3.6 Exercise #3 \end{verbatim}
        \begin{enumerate}[label=\alph*]
            \item 31
            \item 513
            \item 341
            \item 26896
        \end{enumerate}

    \item \begin{verbatim} Ch3.6 Exercise #3 \end{verbatim}
        \begin{enumerate}[label=\alph*]
            \item 2062
            \item 79275
            \item 43962
            \item 233811181
        \end{enumerate}

    \newpage

    \item \begin{verbatim} Ch3.6 Exercise #24 \end{verbatim}
        \begin{enumerate}[label=\alph*]
            \item 1
            \item 1
            \item 1
            \item 139
            \item 1
            \item 1
        \end{enumerate}
    \item \begin{verbatim} Ch3.8 Exercise #2 \end{verbatim}
        \begin{enumerate}[label=\alph*]
            \item 
                \textbf{A} + \textbf{B} = 
                $\begin{bmatrix}
                    0 & 3 & 9 \\
                    1 & 4 & -1 \\
                    2 & -5 & -3    
                \end{bmatrix} $
            \item 
                \textbf{A} + \textbf{B} =
                $\begin{bmatrix}
                    -4 & 9 & 2 & 10 \\
                    -4 & -5 & 4 & 0   
                \end{bmatrix} $
        \end{enumerate}
    \item \begin{verbatim} Ch3.8 Exercise #3 \end{verbatim}
        \begin{enumerate}[label=\alph*]
            \item 
                \textbf{AB} = 
                $\begin{bmatrix}
                    1 & 11 \\
                    2 & 18    
                \end{bmatrix} $
            \item 
                \textbf{AB} =
                $\begin{bmatrix}
                    2 &  -2  & -3 \\
                    1 &   0  & 2 \\
                    9 &  -4  & 4    
                \end{bmatrix} $
            \item 
                \textbf{AB} =
                $\begin{bmatrix}
                    -4 & 15 & -4 & 1\\
                    -3 & 10 & 2 & -3\\
                    0 & 2 & -8 & 6\\
                    1 & -8 & 18 & -13  
                \end{bmatrix} $
        \end{enumerate}
    \item \begin{verbatim} Ch3.8 Exercise #5 \end{verbatim}
        \textbf{A} = $\begin{bmatrix} \frac{9}{5} & -\frac{6}{5} \\ -\frac{1}{5} & \frac{4}{5}  \end{bmatrix}$
    
    \newpage

    \item \begin{verbatim} Ch3.8 Exercise #10 \end{verbatim}
         \begin{enumerate}[label=\alph*]
            \item 3x5
            \item Undefined
            \item 3x4
            \item Undefined
            \item Undefined
            \item 4x5
         \end{enumerate}
    
    \item \begin{verbatim} Ch3.8 Exercise #18  \end{verbatim}
         If a matrix $A^{-1}$ is the inverse of matrix A then the product of the two is the identity matrix.
         Since the product of \newline\newline
         
         $\begin{bmatrix}2 & 3 & -1 \\1 & 2 & 1 \\-1 & -1 & 3\end{bmatrix} $ 
         *
         $\begin{bmatrix}7 & -8 & 5 \\-4 & 5 & -3 \\1 & -1 & 1\end{bmatrix}$
         =
         $\begin{bmatrix}1 & 0 & 0 \\0 & 1 & 0 \\0 & 0 & 1\end{bmatrix}$\newline\newline

         which is a variation of the identity matrix then it can be said that the latter matrix
         is the inverse of the former.

         
        
\end{enumerate}

\end{document}

